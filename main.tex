\documentclass{article}

% \usepackage[margin=1in]{geometry}

% Some color
% \usepackage[usenames,dvipsnames,svgnames,table]{xcolor}

% And hypertext links everywhere
\usepackage{hyperref}

% To get a proper text encoding with xelatex
\usepackage{fontspec}

% Some math packages
\usepackage{amssymb}
\usepackage{amsmath}
\usepackage{bm}
\usepackage{stmaryrd}
\usepackage{mathtools}

% Render inference rules
\usepackage{mathpartir}
% The labels of those rules
\newcommand{\irlabel}[1]{\text{\emph{(#1)}}}

% Used to display stuff like x/y or x//y
\usepackage{xfrac}

% Fore code blocks
\usepackage{listings}
\lstset{%
  escapeinside={//*}{*//}
  }
\lstdefinelanguage{NLight}{%
  morekeywords={%
    let,in
  },%
  morekeywords={[2]Int,true,false,Bool},   % types go here
  otherkeywords={:,=}, % operators go here
  literate={% replace strings with symbols
    {->}{{$\to$}}{2}
    {lambda}{{$\lambda$}}{1}
    {tin}{{$\in$}}{2}
    {\&}{{$\wedge$}}{2}
  },
  basicstyle={\sffamily},
  keywordstyle={\bfseries},
  keywordstyle={[2]\itshape}, % style for types
  keepspaces,
  mathescape % optional
}[keywords,comments,strings]%

\usepackage{todo}

% Tweak the syntax
\usepackage{syntax}
\renewcommand{\grammarlabel}[2]{\meta{#1 #2}}
\newcommand{\meta}[1]{\ensuremath{#1}} % For meta syntax
\renewcommand{\|}{\textrm{|}}

\newcommand{\assign}[2]{\ensuremath{\sfrac{#2}{#1}}}
\newcommand{\assignp} [2] {\assign{#1}{#2}}
\newcommand{\subst} [3] {#3 [\assign{#1}{#2}]}
\newcommand{\substp} [3] {#3 [\assignp{#1}{#2}]}
\newcommand{\dstep} [2] {#1 \ensuremath{\rightarrow} #2}
\newcommand{\ndstep} [2] {#1 \ensuremath{\nrightarrow} #2}
\newcommand{\ndsteps} [2] {#1 \ensuremath{\nrightarrow^*} #2}
\newcommand{\dstepa} [3] {\dstep{#1}{&#2}~\emph{#3} \\}
\newcommand{\eqdef}[2]{#1 \ensuremath{\overset{\text{def}}{=}} #2}
\newcommand{\eqdefa}[3]{\eqdef{#1}{&#2} \emph{#3} \\}

% TODO remove this, it's ugly
\newcommand{\xone}{\ensuremath{x_1}}
\newcommand{\xn}{\ensuremath{x_n}}
\newcommand{\eone}{\ensuremath{e_1}}
\newcommand{\etwo}{\ensuremath{e_2}}
\newcommand{\en}{\ensuremath{e_n}}
% TODO: redefine to get the 0 and 1 of types (with double bar)
\newcommand{\zero}{0}
\newcommand{\one}{1}

% Rendering of symbols and operators
\newcommand{\discrete}[2]{\left\{ #1,\cdots, #2 \right\}}
\newcommand{\ty}[1]{\texttt{#1}}
\newcommand{\set}[1]{\ensuremath{\mathcal{#1}}}
\newcommand{\quasiconst}[1]{\overset{#1}{\twoheadrightarrow}}
\DeclareMathOperator\dom{dom}
\DeclareMathOperator\deff{def}
\DeclareMathOperator\var{\mathcal{V}}
\DeclareMathOperator\A{\mathcal{A}}
\DeclareMathOperator\Bt{\mathcal{B}}
\newcommand{\orthsum}{\oplus^\bot}
\newcommand{\orthplus}{\diamond}
\newcommand{\subtype}{\leq}
\newcommand{\subtypeG}{\precapprox}
\newcommand{\tinfer}{\vdash^\Uparrow}
\newcommand{\tcheck}{\vdash^\Downarrow}
\newcommand{\tIC}{\vdash^\delta}
\newcommand{\onerec}{\{ \textbf{..} \}}
\DeclareCollectionInstance{plainmath}{xfrac}{mathdefault}{math}
{%
  slash-symbol = \sslash{}
}
\newcommand{\ofTypeP}[2]{\UseCollection{xfrac}{plainmath}\sfrac{#1}{#2}}

\newcommand{\Γ}{\Gamma}
\newcommand{\τ}{\ensuremath{\tau}}
\newcommand{\σ}{\sigma}
\DeclareMathOperator\any{\textsc{Any}}
\DeclareMathOperator\grad{\star}
\newcommand{\undef}{\nabla}
\DeclareMathOperator\Int{Int}
\DeclareMathOperator\Bool{Bool}
\newcommand{\λ}{\lambda}
\newcommand{\recleq}{\sqsubsetleq}

% Easier writing of references
\newcommand{\pref}[1]{\ref{#1} at page~\pageref{#1}}

\date{}



\title{A type system for nix}

\begin{document}
\maketitle

\tableofcontents

\pagebreak

In order to lighten the type system, we will not work on the full nix language
(presented in Section~\ref{sec:nix-grammar}), but on a simplified version
called nix-light that we also presents in this document.

We also present a compilation from nix to nix-light.

\section{The nix language}

\subsection{Grammar}
\label{sec:nix-grammar}
The grammar of nix is given at Figures~\pref{grammar::nix}
and~\pref{grammar::types}. It consists\todo{Remove as soon as the grammar is
extended} of a simple lambda calculus with lists (and type annotations).

\begin{figure}
  \begin{grammar}
    \bfseries
    <e> ::=
    $x$ \| $c$
    \alt $\λ$$p$.$e$ \| $e$ $e$
    \alt let $x$ = $e$; $\cdots{}$; $x$ = $e$; in $e$
    \alt [ $e$ $\cdots$ $e$ ]
    \alt if $e$ then $e$ else $e$

    <c> ::= $s$ \| $i$ \| $b$

    <p> ::= $q$ \| $q$@$x$ \| $r$

    <q> ::= ($r$, $r$) \| $q$:\τ

    <r> ::= $x$ | $x$:\τ
  \end{grammar}
  \caption{\label{grammar::nix}The nix grammar for expressions}
\end{figure}

\begin{figure}
  \begin{grammar}
  \bfseries
  <t> ::= $c$ \| $t$ $\bm{\rightarrow}$ $t$
    \alt $t$ $\bm{\vee}$ $t$ \| $t$ $\bm{\wedge}$ $t$ \| $t$ $\bm{\backslash}$ $t$
    \alt [\meta{R}]
    \alt bool \| int \| string

  <u> ::= $t$ \| ?$t$

  <R> ::= $t$ \| \meta{R^{\bm{+}}} \| \meta{R}* \| \meta{R}?
    \| \meta{R} \meta{R} \| \meta{R}\texttt{|}\meta{R}

  <\τ> ::= $t$ % No polymorphism for now
\end{grammar}

  \caption{\label{grammar::types}The nix and nix-light grammar for types}
\end{figure}


\subsection{Typing}
\label{sec:nix-typing}
The typing rules are given by the~\autoref{fig:nix:typing-rules}.
They assume the existence of a function $\Bt$ which assiocates to every constant
$c$ its static type $\Bt(c)$.

They also assume a set $\mathbb{D}$ of builtins functions $f$ and a function
$\mathcal{T}$ from $\mathcal{D}$ to $\textsc{Types}$ such that
\[
  \forall d \in \mathbb{D}, \Bt(d) = (\mathcal{T}(d) \rightarrow \text{True})
                          \cap (\lnot\mathcal{T}(d) \rightarrow \text{False})
\]

Those functions are the functions that checks wether their argument is of a
given type, such as the \lstinline{isInt} function which returns
\lstinline{True} if its argument is an integer, and \lstinline{False} otherwise.


\begin{figure}
  \begin{mathpar}
    \inferrule{ }{\Γ \tinfer x:\Γ(x)}(IVar)
    \and\inferrule{ \Γ(x) \subtypeG \τ }{\Γ; \tcheck x:\τ}(CVar)

    \and\inferrule{ }{\Γ \tinfer c:\Bt(c)}(IConst)
    \and\inferrule{ \Bt(c) \subtypeG \τ }{\Γ \tcheck c:\τ}(CConst)

    \and\inferrule{%
      \Γ \tinfer e_1 : \τ_1 \\ \Γ \tinfer e_2 : \τ_2 \\
      \τ_1 \subtypeG \zero \rightarrow \one \\
      \τ_2 \subtypeG \dom(\τ_1)
    }{%
      \Γ \tinfer e_1~e_2 : \τ_1 \circ \τ_2
    }
    (IApp)

    \and\inferrule{%
      \Γ \tinfer e_2 : \σ \\
      \Γ \tcheck e_1 : \σ \rightarrow \τ
    }{%
      \Γ \tcheck e_1~e_2 : \τ
    }
    (CApp)

    \and\inferrule{%
      \Γ' \dashv p:\accept{p} \\
      \Γ;\Γ' \tinfer e : \τ
    }{%
      \Γ \tinfer \λ p.e : \accept{p} \rightarrow \τ
    }
    (IAbs)

    \and\inferrule{%
      \τ \subtype \zero \rightarrow \one \\
      \forall \σ_1 \rightarrow \σ_2 \in \A(\τ), \\
        \Γ' \dashv p:\σ_1 \\ \Γ;\Γ' \tcheck e : \σ_2
    }{%
      \Γ \tcheck \λ p.e : \τ
    }(CAbs)

    \and
    \inferrule{%
      \forall i \in \discrete{1}{n},
        \Γ; x_1 : ?; \cdots; x_n : ? \tinfer e_i : \τ_i \\
      \Γ; x_1 : \τ_1; \cdots; x_n : \τ_n \tIC e : \τ
    }{%
      \Gamma \tIC \text{let } x_1 = e_1; \cdots; x_n = e_n
        \text{ in } e : \τ
    }
    (Let)

    \and
    \inferrule{%
      \forall i \in \discrete{1}{n},
        \Γ; x_1 : \τ_1; \ldots; x_n : \τ_n \tcheck e_i : \τ_i \\
      \Γ; x_1 : \τ_1; \cdots; x_n : \τ_n \tIC e : \τ
    }{%
      \Gamma \tIC \text{let } x_1 : \τ_1 = e_1; \ldots{}; x_n : \τ_n = e_n
        \text{ in } e_0 : \τ_0
    }
    (LetAnnot)

    \and\inferrule{%
      f \in \mathbb{D} \\
      t = \mathcal{T}(f) \\
      \Γ \tinfer e : \τ \\
      \τ \not\subtype t \Rightarrow \Γ; x : \τ \wedge \lnot t \tinfer e_2 : \τ_2 \\
      \τ \not\subtype \lnot t \Rightarrow \Γ; x : \τ \wedge t \tinfer e_1 : \τ_1 \\
    }{%
      \Γ \tinfer \text{ if } f~x \text{ then } e_1 \text{ else } e_2 : \τ_1 \cup \τ_2
    }(ITcase)

    \and\inferrule{%
      f \in \mathbb{D} \\
      t = \mathcal{T}(f) \\
      \Γ \tinfer e : \τ \\
      \τ \not\subtype t \Rightarrow \Γ; x : \τ \wedge \lnot t \tcheck e_2 : \σ \\
      \τ \not\subtype \lnot t \Rightarrow \Γ; x : \τ \wedge t \tcheck e_1 : \σ \\
    }{%
      \Γ \tcheck \text{ if } f~x \text{ then } e_1 \text{ else } e_2: \σ
    }(CTcase)

    \and\inferrule{%
      \Γ \tinfer e : \τ \\ \τ \subtype \text{Bool} \\
      \Γ \tinfer e_1 : \τ_1 \\
      \Γ \tinfer e_2 : \τ_1
    }{%
      \Γ \tinfer \text{ if } f~x \text{ then } e_1 \text{ else } e_2 : \τ_1 \cup \τ_2
    }(IIf)

    \and\inferrule{%
      \Γ \tinfer e : \τ \\ \τ \subtype \text{Bool} \\
      \Γ \tcheck e_1 : \σ \\
      \Γ \tcheck e_2 : \σ
    }{%
      \Γ \tcheck \text{ if } f~x \text{ then } e_1 \text{ else } e_2 : \σ
    }(CIf)

    \and\inferrule{%
      \forall i \in \discrete{1}{n}, \Γ \tIC e_i : \τ_i
    }{%
      \Γ \tIC [ e_1 \cdots e_n ] : [ \τ_1 \cdots \τ_n ]
    }(List)
  \end{mathpar}
  \caption{Typing rules for nix}\label{fig:nix:typing-rules}
\end{figure}


\subsection{Preprocessing}
\todo{Reformulate to fit the position in the document}
Although the nix language isn't a very complex one, it still contains a lot of
syntactic sugar that we want to get rid of, and because of its flexibility is
really hard to type − because a lot of patterns that are syntactic in other
languages are only determined by the semantics, and thus impossible to detect
statically.

In particular, most statically typed languages which have a notion of type at
runtime have a special syntactic construct to do case analysis on the type of
a variable. For example, in the CDuce language~\cite{Fri04}, this case
analysis is expressed by a special form of pattern that can be used in
pattern-matching. This is important because it means that those informations
can be used by the type-checker. In Nix, however, this case analysis is done by
the combination of if-then-else's constructs and functions that tell wether
their argument is of a certain type (\lstinline{isInt} which returns
\lstinline{true} if its argument is an integer for example).
The problem is that some expressions that are reasonable can't be typed without
enriching the typing environment in one of the branches of the if. For example,
one may expect the expression \lstinline{if isInt $x$ then $x + 1$ else $x$} to
be well typed, but this requires the type-checker to be aware that in the first
branch $x$ is of type \lstinline{Int}.

To work around this problem, we restrict ourselves to some syntactic patterns
that we can recognise. In practice, to keep the type system simple, we only try
to enrich the environment in presence of an expression of the form
\lstinline{if isT $x$ then $e_1$ else $e_2$}.

As this is rather limiting, we allow ourselves a pre-processing phase that
converts some slightly more complicated expressions into the pattern above −
while preserving the semantics of course.
The list of those conversions may be expanded at will − that's why it is kept
out of the type system, but the figure~\pref{semantics::pre-processing}
provides a few of them. (Although we did not give any formal semantics for nix
yet, we can at least say that those intuitively preserve the expected semantics
of a if-then-else construct).

\begin{figure}
  \begin{align}
    \text{if $e'_1$ || $e'_2$ then $e_1$ else $e_2$} &\rightarrow
      \text{if $e'_1$ then $e_1$ else (if $e'_2$ then $e_1$ else $e_2$)} \\
    \text{if $e'_1$ \&\& $e'_2$ then $e_1$ else $e_2$} &\rightarrow
      \text{if $e'_1$ then (if $e'_2$ then $e_1$ else $e_2)$ else $e_2$} \\
    \text{if not $e$ then $e_1$ else $e_2$} &\rightarrow
      \text{if $e$ then $e_2$ else $e_1$}
  \end{align}
  \caption{Simple pre-processing phase for nix expressions}\label{semantics::pre-processing}
\end{figure}


\section{Nix-light}

\subsection{Grammar}
\label{sec:nix-light-grammar}
The grammar of nix-light is based of the grammar of nix and brings several
modifications:
\begin{itemize}
  \item A first huge change is the removal of the \emph{if} construct which is
    replaced by a more general \emph{typecase} which is easier to reason on.

  \item Another notable change is that the opaque list construct of nix is
    replaced by the classical \texttt{nil} and \texttt{cons}.
    This avoids having over-complicated typing and evaluation rules for lists.

    For consistency, a pattern for lists has also been added.
\end{itemize}

The grammar of nix-light is given in the
figures~\pref{grammar::expressions},~\pref{grammar::values}
and~\pref{grammar::types} (the types are the same as nix types).

\begin{figure}
  \begin{grammar}
  \bfseries
  <e> ::=
    $x$ \| $c$
    \alt $e$.$a$ \| $e$.$a$ or $e$
    \alt $\λ$$p$.$e$ \| $e$ $e$
    \alt let $x$ = $e$; $\cdots{}$; $x$ = $e$; in $e$
    \alt let $x$ : $\τ$ = $e$; $\cdots{}$; $x$ : $\τ$ = $e$; in $e$
    \alt Cons ($e$, $e$)
    \alt ($x$ = $e$ $\bm{\in}$ $t$) ? $e$ : $e$

  <c> ::= $s$ \| $i$ \| $b$ \| Nil

  <p> ::= $q$ \| $q$@$x$ \| $r$

  <q> ::= Cons ($r$, $r$) \| Nil \| $q$:\τ

  <r> ::= $x$ | $x$:\τ
\end{grammar}

  \caption{\label{grammar::expressions}The nix-light grammar for expressions}
\end{figure}

\begin{figure}
  \begin{grammar}
  \bfseries

  <v> ::=
    $c$
    \alt $\λ p$.$e$
    \alt Cons ($e$, $e$) \| Nil
\end{grammar}

  \caption{\label{grammar::values}The nix-light grammar for values}
\end{figure}


\subsection{Semantics}
\begin{figure}
  \center
  \def\leadsto{\ensuremath{\rightsquigarrow}}
  \begin{tabular}{rl}
  °(λx.e1) e2° &\leadsto °e1[x := e2]° \\
  °(λp.e) v° &\leadsto °e[$\sfrac{p}{v}$]° \\
  °(x = v tin t) ? e1 : e2° &\leadsto °e1[x := v]° \quad if $\vdash v : t$ \\
  °(x = v tin t) ? e1 : e2° &\leadsto °e2[x := v]° \quad if $\vdash v : \lnot t$ \\
  °{ x = e; $\cdots$ }.x° &\leadsto °e° \\
  °{ x = e; $\cdots$ }.x or e'° &\leadsto °e° \\
  °{ x1 = e1; $\cdots$; xn = en }.x or e'° &\leadsto °e'°
      \quad if $x \notin \left\{ x1, \cdots, xn \right\}$ \\
  °e : τ° &\leadsto °e°
  \end{tabular}
  \caption{Nix-light operational semantics\label{nix-light::semantics}}
\end{figure}


\section{Nix-light typing}

\subsection{The $\λ\&-calculus$}
\subsubsection{Patterns}

In order to type patterns, we need to introduce a new form of typing judgement.
The judgement $\Γ \dashv p:\τ$ means that when applied against a type $\τ$, the
pattern $p$ will enrich the environment with the constraints $\Γ$.

For example, we got $x:\Int \dashv x:\Int$, which reads ``If we apply a term
of type $\Int$ to the pattern $x$, then the environment on the
right of the pattern will be enriched with the constraint $x:\Int$''

Likewise, the following statement holds.
\[x:\Int; y: \Bool \dashv \left\{ x; y ? \text{true}; \right\} : \{ x = \Int; y =? \Bool \}\]
This means that when if we match a term of type $\left\{ x =\Int; y =? \Bool
\right\}$ against the pattern $\left\{ x; y ? \text{true}; \right\}$, then the
environment on the right side of the pattern will be enriched with the
constraints $x : \Int$ and $y : \Bool$.

As the symbol ``$\dashv$'' suggests, this typing judgement is the converse of
the classical typing judgement $\Γ \vdash e : \τ$ for expressions: instead of
stating that under the hypothesis $\Γ$, the expression $e$ has type $\τ$, we
state that if the pattern $p$ has type $\τ$, then in produces the environment
$\Γ$.

The typing rules for this statement are given by the
figure~\pref{typing::patterns::typing-rules}.

Maybe~\todo{Find out wether this is true} this enjoys principal typing.

\todo{Define the projections $\pi_1$ and $\pi_2$}.

\begin{figure}
  \begin{mathpar}
  \inferrule{ }{\Γ \tinfer x:\Γ(x)}(IVar)
  \and\inferrule{ \Γ(x) \subtypeG \τ }{\Γ; \tcheck x:\τ}(CVar)

  \and\inferrule{ }{\Γ \tinfer c:\Bt(c)}(IConst)
  \and\inferrule{ \Bt(c) \subtypeG \τ }{\Γ \tcheck c:\τ}(CConst)

  \and\inferrule{%
    \Γ \tinfer e_1 : \τ_1 \\ \Γ \tinfer e_2 : \τ_2 \\
    \τ_1 \subtypeG \zero \rightarrow \one \\
    \τ_2 \subtypeG \dom(\τ_1)
  }{%
    \Γ \tinfer e_1~e_2 : \τ_1 \circ \τ_2
  }
  (IApp)

  \and\inferrule{%
    \Γ \tinfer e_2 : \σ \\
    \Γ \tcheck e_1 : \σ \rightarrow \τ
  }{%
    \Γ \tcheck e_1~e_2 : \τ
  }
  (CApp)

  \and\inferrule{%
    \Γ' \dashv p:\accept{p} \\ % Replace by the type accepted by p
    \Γ;\Γ' \tinfer e : \τ
  }{%
    \Γ \tinfer \λ p.e : \accept{p} \rightarrow \τ
  }
  (IAbs)

  \and\inferrule{%
    \τ \subtype \zero \rightarrow \one \\
    \forall \σ_1 \rightarrow \σ_2 \in \A(\τ), \\
      \Γ' \dashv p:\σ_1 \\ \Γ;\Γ' \tcheck e : \σ_2
  }{%
    \Γ \tcheck \λ p.e : \τ
  }(CAbs)

  \and
  \inferrule{%
    \forall i \in \discrete{1}{n},
      \Γ; x_1 : ?; \cdots; x_n : ? \tinfer e_i : \τ_i \\
    \Γ; x_1 : \τ_1; \cdots; x_n : \τ_n \tIC e : \τ
  }{%
    \Gamma \tIC \text{let } x_1 = e_1; \cdots; x_n = e_n
      \text{ in } e : \τ
  }
  (Let)

  \and
  \inferrule{%
    \forall i \in \discrete{1}{n},
      \Γ; x_1 : \τ_1; \ldots; x_n : \τ_n \tcheck e_i : \τ_i \\
    \Γ; x_1 : \τ_1; \cdots; x_n : \τ_n \tIC e : \τ
  }{%
    \Gamma \tIC \text{let } x_1 : \τ_1 = e_1; \ldots{}; x_n : \τ_n = e_n
      \text{ in } e_0 : \τ_0
  }
  (LetAnnot)

  \and\inferrule{%
    \Γ \tinfer e : \τ \\
    \τ \not\subtype t \Rightarrow \Γ; x : \τ \wedge \lnot t \tinfer e_2 : \τ_2 \\
    \τ \not\subtype \lnot t \Rightarrow \Γ; x : \τ \wedge t \tinfer e_1 : \τ_1 \\
  }{%
    \Γ \tinfer (x = e \in t) ? e_1 : e_2 : \τ_1 \vee \τ_2
  }(ITcase)

  \and\inferrule{%
    \Γ \tinfer e : \τ \\
    \τ \not\subtype t \Rightarrow \Γ; x : \τ \wedge \lnot t \tcheck e_2 : \σ \\
    \τ \not\subtype \lnot t \Rightarrow \Γ; x : \τ \wedge t \tcheck e_1 : \σ \\
  }{%
    \Γ \tcheck (x = e \in t) ? e_1 : e_2 : \σ
  }(CTcase)

  \and\inferrule{%
    \Γ \tIC e_1 : \τ_1 \\ \Γ \tIC e_2 : \τ_2 \\
    \τ_2 \subtypeG \cons(\any, \any)
  }{%
    \Γ \tIC \cons(e_1, e_2) : \cons(\τ_1, \τ_2)
  }(Cons)
\end{mathpar}

  \caption{Typing rules for the $\λ\&-calculus$\label{typing::lambda-calculus}\\
  \small{The ``$\delta$'' symbol means either ``$\Uparrow$'' either ``$\Downarrow$''
  (but always the same within a given inferrence rule)}}
\end{figure}

 \begin{figure}
  \begin{center}
    \begin{align*}
      \eqdefa{\accept{x}}{\grad}{}
      \eqdefa{\accept{x:\τ}}{\τ}{}
      \eqdefa{\accept{q@x}}{\accept{q}}{}
      \eqdefa{\accept{\cons(x_1, x_2)}}{\cons(\accept{x_1}, \accept{x_2})}{}
      \eqdefa{\accept{\cons(x_1, x_2) : \τ}}{%
        \accept{\cons(x_1 : \pi_1(\τ), x_2 : \pi_2(\τ))}
      }{\text{If }\τ \subtypeG Cons(\one, \one)}
      \eqdefa{\accept{\nil}}{\nil}{}
    \end{align*}
  \end{center}\label{typing::pattern-accept}
   \caption{Semantics of the $\accept{\_}$ operator}
\end{figure}
\begin{figure}
  \begin{mathpar}
    \inferrule{~}{x:\τ \dashv x:\τ}
    \and\inferrule{~}{\dashv \text{Nil} : \text{Nil}}
    \and\inferrule{\Γ \dashv p:\τ \\ \τ \subtypeG \σ}{\Γ \dashv (p:\σ):\τ}
    \and\inferrule{%
      \τ \subtypeG \cons(\one,\one) \\
      \Γ_1 \dashv p_1 : \pi_1(\τ) \\
      \Γ_2 \dashv p_2 : \pi_2(\τ) \\
      \text{Vars}(\Γ_1) \cap \text{Vars}(\Γ_2) = \varnothing
    }{%
      \Γ_1; \Γ_2 \dashv \cons(p_1, p_2) : \τ
    }
    \and\inferrule{%
      \Γ \dashv p : \τ \\
      x \notin \Γ
    }{%
      \Γ; x:\τ \dashv p@x : \τ
    }
  \end{mathpar}
  \caption{Typing rules for the patterns\label{typing::patterns::typing-rules}}
\end{figure}

\subsubsection{Typecase}

The typecase $(x := e \in t) ? e_1 : e_2$ can be typed in a simple way, by
saying that if $e$ has a type $τ$, $e_1$ has type $\τ_1$ and $e_2$ has type
$\τ_2$ (under the current typing environment $\Γ$), then $(x := e \in t) ? e_1
: e_2$ has type $\τ_1 \vee \τ_2$.
However, doing so means that we do not use the extra type information given by
``$e \in t$'', which loosens a lot the interest of this construct. For example,
an expression such as $(x := e \in \bm{{Int}}) ? x + 1 : x$, with $\vdash e :
\any$ wouldn't typecheck, as $x+1$ isn't well typed without any further
constraint on the type of $x$.

A more interesting typing rule would state that if $\Γ; x:\τ \wedge t \vdash
e_1: \τ_1$ and $\Γ; x:\τ \wedge \lnot t \vdash e_2: \τ_2$ (where $\τ$ is a type
of $e$ under the hypothesis $\Γ$), then the whole expression has type $\τ_1
\vee \τ_2$.
With this rule, the expression $(x := e \in \bm{{Int}}) ? x + 1 : x$ is
well-typed (provided that $e$ is).

The typing rules are given by the
figures~\pref{typing::lambda-calculus},~\pref{typing::records}
and~\pref{typing::operators}.


\subsection{Bidirectional typing}
\subsubsection{Motivation and overview}
\label{motivation-and-overview}

The rules defined above are already quite expressive. However, they aren't
enough to type the following function, no matter how the type annotations are
written:

\begin{lstlisting}[language=NLight]
  (lambda cond. lambda x.  (y := cond in true) ? x+1 : not x
  : (true -> Int -> Int & false -> Bool -> Bool))
\end{lstlisting}

Indeed, do to this we would need to annotate $x$ as \ty{Int} or \ty{Bool}
depending on wether \texttt{cond} was \ty{true} or \ty{false}, which isn't
possible.

To remedy this problem, we split our type system in two parts: an inference
part and a checking part.  The inference part − denoted with typing judgements
of the form $\Γ \tinfer e : \τ$ − corresponds to classical bottom-up
type-inference, while the checking part − denoted with typing judgements of the
form $\Γ \tcheck e : \τ$ corresponds to a top-down type inference, where the
type of the expression is already known and we use it to infer the type of the
sub-expressions − in other words, we propagate the type annotations to the
bottom while type-checking. In particular, this ``checking'' type-system allows
us a more precise typing of lambdas.

Explicitly annotated let-bindings are thus typed using the checking type-system
and we can rewrite the previous expression as
\begin{lstlisting}[language=NLight]
let f : (true -> Int -> Int & false -> Bool -> Bool) =
 lambda cond. lambda x.  (y := cond tin true) ? x+1 : not x
in f
\end{lstlisting}

In this case, we just have to \emph{check} that
\begin{lstlisting}[language=NLight]
lambda x. (y := cond tin true) ? x+1 : not x
\end{lstlisting}
has type $\ty{Int} \rightarrow \ty{Int}$ under the hypothesis $\texttt{cond} :
\ty{true}$ and $\ty{Bool} \rightarrow \ty{Bool}$ under the hypothesis
$\texttt{cond} : \ty{false}$. This means checking that
\begin{lstlisting}[language=NLight]
(y := cond tin true) ? x+1 : not x
\end{lstlisting}
has type $\ty{Int}$ under the hypothesis $\texttt{cond} : \ty{true}; \texttt{x}
: \ty{Int}$ and $\ty{Bool}$ under the hypothesis $\texttt{cond} : \ty{false};
\texttt{x} : Bool$. This is true, thanks to the ruls CTcase of the type system
(figure~\pref{typing::lambda-calculus}).

So the basic idea behind this is to use the type annotations of parent nodes
(or any other type information that we got from upper in the AST), not just to
check that the inferred type was correct, but in the inference process itself.

For example, if we want to type the expression $\cons (e_1, e_2) : \cons (\τ_1,
\τ_2)$, we will try to type $(e_1 : \τ_1)$ and $(e_1 : \τ_2)$ and then merge
the results~\footnote{It is possible to extend this to a more general case
where the type annotation is any subtype $\τ$ of $\cons(\any, \any)$ using a
\emph{maximal product decomposition} as defined by Kim Nguyễn in~\cite{phdkim},
but this out of our scope for now.}.

\subsubsection{Arrows}

Another construct for which we want to propagate type informations is the
definition of a function.

Assume we got an expression $(\λ p . e) : \τ$ where $\τ$ is a subtype (not
gradual) of $\zero \rightarrow \one$.

We want to type the function for each concrete arrow type included in $\τ$. In
other words, if we note $\A(\τ)$ the set of all arrow types in $\τ$, we want that
for all $\σ \rightarrow \σ' \in \A(\τ)$, $p$ matches $\σ$, and that under this
matching, $e$ has type $\σ'$.

This is given by the rule \emph{CAbs} of the type system in the
Figure~\pref{typing::lambda-calculus}.

Remains the definition of $\A(\τ)$, that we give as follows:

If $\τ$ is in the form
\[
  \τ = \bigvee\limits_{i\in I}\left(
    \bigwedge\limits_{p\in P_i} (\σ_p \rightarrow \τ_p)
    \wedge \bigwedge\limits_{n \in N_i} \lnot (\σ_n \rightarrow \τ_n)
  \right)
\]
then $\A(\τ)$ is given by:
\[
  \A(\τ) = \bigsqcup\limits_{i \in I} \{ \σ_p \rightarrow \τ_p \| p \in P_i \}
\]
where $\sqcup$ is defined as
\[
  \{ \σ_i \rightarrow \τ_i \| i \in I \} \sqcup \{ \σ_j \rightarrow \τ_j \| j \in J \} =
    \{ (\σ_i \wedge \σ_j) \rightarrow (\τ_i \vee \τ_j) \| i \in I, j \in J \}
\]

In the example of the Section~\ref{motivation-and-overview}, the expression has
type $t = (true \rightarrow (Int \rightarrow Int)) \wedge (false \rightarrow
(Bool \rightarrow Bool))$

Thus, we have $\A(t)$ equal to the set $\left\{ true \rightarrow (Int
\rightarrow Int); false \rightarrow (Bool \rightarrow Bool) \right\}$


\subsection{Soundness}
We have the two classical results of \emph{Subject reduction} and
\emph{Progress} which entail type soundess for the non-gradual part of the
language.
Note that because we defined two typing judgements, both result will be stated
twice.

We first prove the following lemmas:

\begin{lemma}\label{lemma:inferCheck}
  Let $e$ be an expression, $\Γ$ a typing environment and $\τ$ and $\τ'$ two
  (possibly gradual) types with $\τ \subtypeG \τ'$.
  If $\Γ \tinfer e : \τ$ then $\Γ \tcheck e : \τ'$.
\end{lemma}

\begin{proof}
  \todo{}
\end{proof}

\begin{lemma}[Substitution]\label{lemma:substitution}
  Let $e$ and $e'$ be expressions, $x$ be a variable, $\τ$ and $\τ'$ two types,
  $\Γ$ a typing environment and $\delta$ be $\Uparrow$ or $\Downarrow$.

  If $\Γ; x : \τ' \tIC e : \τ$ and $\Γ \tinfer e' : \τ'$ then $\Γ \tIC
  \subst{x}{e'}{e} : \τ$
\end{lemma}

\begin{proof}
  By induction on the typing derivation of $\Γ; x : \τ' \tIC e : \τ$. We
  replace every Ivar rule introducing $\Γ ; x : \τ' \tinfer x : \τ'$ by a
  derivation of $\Γ \tinfer e' : \τ'$, and every Cvar rule introducing $\Γ ; x
  : \τ' \tcheck x : \τ''$ (with $ \τ' \subtypeG \τ'')$ by a derivation of $\Γ
  \tcheck e' : \τ''$ (which exists because of Lemma~\ref{lemma:inferCheck}).

  This builds a new derivation of $\Γ \tinfer \subst{x}{e'}{e} : \τ$.
\end{proof}

\begin{theorem}[Subject reduction $\Uparrow$]\label{thm:subj-reduction-infer}
  For any pair $e, e'$ of terms (of nix-light), if $\Γ \tinfer e : t$ and $e
  \rightarrow e'$, then $\Γ \tinfer e' : t$.
\end{theorem}

\begin{proof}
  We consider an expression $e$ such that $\Γ \tinfer e : t$.
  We prove by induction on the derivation of $\Γ \tinfer e : t$ that $\forall
  e', (e \rightarrow e') \Rightarrow (\Γ \tinfer e' : t)$.

  Let's consider the various possibilities for the last rule of the derivation
  $\Γ \tinfer e : t$.

  \begin{description}
    \item[IVar,IConst,IAbs,ICons] The expression $e$ is a value and can't be
      reduced, so the property holds.
    \item[IApp]
      $\inferrule{%
        \Γ \tinfer e_1 : t_1 \\ \Γ \tinfer e_2 : t_2 \\
        t_1 \subtypeG \zero \rightarrow \one \\
        t_2 \subtypeG \dom(t_1)
      }{%
        \Γ \tinfer e_1~e_2 : t_1 \circ t_2
      }$

      The expression $e$ has then the form $e_1~e_2$ with $\Γ \tinfer e_1
      : t_1$ and $\Γ \tinfer e_2 : t_2$ (and $t = t_1 \circ t_2$).
      It can be reduced in three different ways (depending of the form of $e_1$
      and $e_2$):
      \begin{itemize}
        \item If $e_1$ is a value $\λ r. e_0$, then the only possible reduction
          is by applying the $\beta$-reduction rule, so the only choice for
          $e'$ is $\subst{x}{e_2}{e_0}$ (where $x = \var(r)$).

          Moreover, a case analysis on the different typing rules shows that
          the last rule of the derivation of $\Γ \tinfer \λ r . e_0 : t_1$ can
          only be the Iabs rule:
          \[
            \inferrule{%
              x:t_x \vdash r:t_x \\ \Γ; x:t_x \tinfer e_0:s
            }{%
              \Γ \tinfer \λ r.e_0 : t_x \rightarrow s
            }
          \]
          So $t_1$ has the form $t_x \rightarrow s$. This means that $t_2$ is a
          subtype of $t_x = \dom(t_x \rightarrow s)$ and that $t'$ is equal to
          $s = (t_x \rightarrow s) \circ t_2$.

          Moreover, as $\Γ; x:t_x \tinfer e_0:s$ and $\Γ \tinfer e_2 : t_2$
          with $t_2 \subtypeG t_x$, the Lemma~\ref{lemma:substitution} allows
          us to conclude that $\Γ \tinfer e' : t'$.
        \item If $e_1$ is a value $\λ p.e_0$ (with $p$ not in the form $x$ or
          $x:t$), and $e_2$ is a value, then we can use the same reasoning.
        \item In the other cases, we can reduce either $e_1$, either $e_2$.

          If we reduce $e_1$ to $e'_1$, we get a new expression $e' =
          e'_1~e_2$, and $e'_1$ satisfies $\Γ \tinfer e'_1 : t_1$ (by induction
          hypothesis). By re-applying the Iapp rule, we get $\Γ \tinfer
          e'_1~e_2 : t$, thus $\Γ \tinfer e' : t$.

          The same holds if we reduce $e_2$.

      \end{itemize}
      \item[Let]
        $\inferrule{%
          \Γ; x_1 : \any \tinfer e_1 : t_1\\
          \Γ ; x_1 : t_1 \tinfer e_2 : t
        }{%
          \Γ \tinfer \text{let } e_1 = x_1 \text{ in } e_2 : t
        }$

        The expression $e$ has the form ``let $e_1 = x_1$ in $e_2$''.
        It reduces (and may only reduce to) $e' = \subst{x_1}{e_1}{e_2}$.
        By applying the substitiution lemma, we can deduce that $e'$ has type
        $t$.
      \item[LetAnnot]
        The subject reduction property do not hold for the \textbf{LetAnnot}
        case for now.~\todo{Make this hold}
      \item[ITcase]
        $\inferrule{%
          \Γ \tinfer e_0 : t_0 \\
          t_0 \not\subtype s \Rightarrow \Γ; x : t_0 \wedge \lnot s \tinfer e_2 : t_2 \\
          t_0 \not\subtype \lnot s \Rightarrow \Γ; x : t_0 \wedge s \tinfer e_1 : t_1 \\
        }{%
          \Γ \tinfer (x = e_0 \in s) ? e_1 : e_2 : t_1 \vee t_2
        }$

        If $e_0$ is not a value, then the only possible reduction for $e$ is to
        reduce it to an expression $e'_0$ which will have a type $t'_0 \subtype
        t_0$ under the context $\Γ$ (by induction hypothesis).
        By re-applying the \textbf{ITcase} rule, we obtain a new type $t'$ for
        the expression $(x = e'_0 \in s) ? e_1 : e_2$, which is a subtype of
        $t$.

        If $e_0$ is a value $v$, then the only possible reductions are if
        $\tinfer v : s$ or $\tinfer v : \lnot s$\todo{Current typing rules
        do not allow asserting that this is always true, which will b a deal
        broker for Progress, see~\cite{Fri08} for a way to enforce this}.

        Assume the first one. The expression $e$ then reduces to
        $\subst{x}{v}{e_1}$, and (by application of the substitution Lemma), we
        know that $\Γ \tinfer \subst{x}{v}{e_1} : t_1$. As $t_1 \subtype t$
        (because $t = t_1 \vee t_2$), we get the expected result.

        By symmetry, this also holds if $\tinfer v : \lnot s$.
  \end{description}
\end{proof}

\begin{theorem}[Subject reduction $\Downarrow$]\label{thm:subj-reduction-check}
  For any pair $e, e'$ of terms (of nix-light), if $\Γ \tcheck e : t$ and $e
  \rightarrow e'$, then $\Γ \tcheck e' : t$.
\end{theorem}

\begin{proof}
  The proof is similar to the one of~\autoref{thm:subj-reduction-infer}.
\end{proof}


\section{Compilation}

\subsection{Compilation rules}
Both languages are rather similar, so the compilation is mostly
straightforward: nix lambdas are compiled to nix-light lambdas; nix constants
to nix-light constants, and so on\ldots. The intersesting parts are:

\begin{description}
  \item[If-then-else's] Those are compiled, to typecases, as said in
    Section~\ref{sec:nix-light-grammar}. However, there are several
    possibilities for this.

    The obvious one is to translate \lstinline{if $e$ then $e_1$ else $e_2$} into
    \lstinline{($x$ := ($e$ : Bool) tin true) ? $e_1$ : $e_2$} (where $x$ do not appear
    free in $e_1$ or $e_2$).
    However, the interest of the typecase is that it gives us some information
    on the type of $x$, that we can't use at all here. So this translation
    isn't very interesting.
    When possible, we will  to define smarter compilations for some forms of
    if-then-else.

    The most notable pattern that we want to recognize is
    \lstinline{if isT $x$ then $e_1$ else $e_2$}
    (where \lstinline{isT} is a predicate on types such as
    \lstinline{isInt}, \lstinline{isBool}, \ldots).
    Such an expression will be compiled to a typecase of the form
    \lstinline{($x$ := $x$ tin T) ? $e_1$ : $e_2$}, which will later allow us to take
    advantage of the knowledge of the type of $x$.

  \item[Lists] Nix opaque lists are replaced in nix-light by the more
    conventional algebraic datatype built using the \lstinline{Cons} and
    \lstinline{Nil} constructors.
    This translation is rather easy: the compilation simply considers that
    \lstinline{[ $e_1$ $\cdots$ $e_n$ ]} is syntactic sugar for
    \lstinline{Cons($e_1$, Cons($\cdots$, Cons($e_n$, Nil)))} %chktex 36
\end{description}

The exact semantics of nix in term of nix-light are given by the
Figure~\pref{compilation}.

\begin{figure}
  \begin{align*}
    \sem{x} &= x \\
    \sem{c} &= c \\
    \sem{\λ p.e} &= \λ p. \sem{e} \\
    \sem{e_1\ e_2} &= \sem{e_1}\ \sem{e_2} \\
    \sem{\text{let } x_1 = e_1; \cdots; x_n = e_n; \text{ in } e} &=
      \text{let } x_1 = \sem{e_1}; \cdots; x_n = \sem{e_n};
        \text{ in } \sem{e} \\
    \sem{[e_1\ e_2\ \cdots\ e_n ]} &=
      \text{Cons}(\sem{e_1},\ \sem{[e_2\ \cdots\ e_n]}) \\
    \sem{[ e ]} &= \text{Cons}(\sem{e}, \text{Nil}) \\
    \sem{\text{if isT $x$ then $e_1$ else $e_2$}} &=
      (x := x \in T) ? e_1 : e_2 \\
    \sem{\text{if $e$ then $e_1$ else $e_2$}} &=
      (x := (\sem{e} : \text{Bool}) \in \text{true})\ ?\ \sem{e_1} : \sem{e_2} \\
      & \text{ where $e$ isn't of the form ``isT $x$''}
  \end{align*}
  \caption{Semantics of the compilation from nix to nix-light}\label{compilation}
\end{figure}


\subsection{Preservation of typing}

\todos{}

\bibliographystyle{alpha}
\bibliography{references}
\end{document}
