\documentclass[pagesize=false,frenchb]{scrartcl}
% \usepackage[margin=1in]{geometry}

\usepackage[usenames,dvipsnames,svgnames,table]{xcolor}

% Put some french in it
\usepackage{babel}

\newcommand{\pref}[1]{\ref{#1} à la page~\pageref{#1}}

\usepackage[toc,page]{appendix}

% And hypertext links everywhere
\usepackage{hyperref}

% To get a proper text encoding with xelatex
\usepackage{fontspec}
\setmainfont{Crimson-Roman.otf}[
  Path =./fonts/,
  BoldFont = Crimson-Bold.otf,
  ItalicFont = Crimson-Italic.otf,
  BoldItalicFont = Crimson-BoldItalic.otf
]

% Render inference rules
\usepackage{mathpartir}
% The labels of those rules
\newcommand{\irlabel}[1]{\text{\emph{(#1)}}}

% Fore code blocks
\usepackage{listings}
\definecolor{lstbg}{rgb}{0, 0, 0.3}
\lstdefinelanguage{NLight}{%
  morekeywords={%
    let,in,if,then,else,Cons,Nil
  },%
  morekeywords={[2]Int,true,false,Bool,T},   % types go here
  otherkeywords={:,=[]}, % operators go here
  literate={% replace strings with symbols
    {->}{{$\to$}}{2}
    {lambda}{{$\lambda$}}{1}
    {tin}{{$\in$}}{2}
    {\&}{{$\wedge$}}{2}
  },
  basicstyle={\sffamily},
  keywordstyle={\bfseries},
  keywordstyle={[2]\itshape}, % style for types
  keepspaces,
  mathescape % optional
}[keywords,comments,strings]
\lstset{%
  escapeinside={//*}{*//},
  breaklines=true,
  mathescape=true,
  language=NLight,
  basicstyle=\color{lstbg}
}
\lstMakeShortInline{°}

\usepackage{syntax}

\newcommand{\τ}{\ensuremath{\tau}}

\title{Rapport de stage}
\subtitle{Un système de types pour Nix}
\author{Théophane Hufschmitt}
\date{21 Août 2017}

\begin{document}
\maketitle

\tableofcontents

\pagebreak

\section*{Abstract}

\section{Contexte}
% État de l'art, motivation du stage

\subsection{Nix}
% Description de Nix et de toutes les horreurs qu'il contient
% Explication rapide de ce qui est nécessaire pour le typer à peu près
% raisonnablement

\subsubsection{Présentation générale}
% Explication de son utilisation et justification de la volonté de le typer
\paragraph{Le gestionnaire de paquet :}

Nix \cite{phdeelco} est un gestionnaire de paquets pour les systèmes Unix visant à appliquer à
la gestion de paquets des concepts provenant du monde des langages de
programmation (fonctionnelle en particulier).

Les gestionnaires de paquets ``traditionnels'' (\verb|rpm|, \verb|apt-get|)
manipulent le système de fichier en place, installer ou désinstaller un paquet
revenant à effectuer une modification dessus. De telles modifications sont
extrêmement difficiles à contrôler, d'autant plus que l'installation des
paquets est souvent en partie reléguée à des scripts shell dont la sémantique
est connue pour être extrêmement complexe.
% TODO: Référence au travail de Ralf Treinen & cie
Du point de vue du programmeur, ce mode de fonctionnement revient à avoir un
état global mutable, ce qui est généralement considéré comme une pratique
dangereuse.
Dans la pratique, ce modèle pose de nombreux problèmes. Par exemple, une mise à
jour interrompue (coupure de courant, interruption par l'utilisateur) peut
laisser le système dans un état incohérent car les opérations ne sont pas
atomiques.
De plus, l'état d'un système est très difficilement reproductible: deux
machines avec le même ensemble de paquets installés, mais dans un ordre
différent par exemple peuvent être dans des états totalement différents (pire:
installer puis désinstaller un paquet laisse souvent le système dans un état
différent de l'état de départ).

Nix propose une approche radicalement différente : Du point de vue de
l'utilisateur, la configuration du système est entièrement décrite par le
résultat de l'évaluation d'une expression dans un langage fonctionnel pur
(appelé lui aussi Nix). Ce mode de fonctionnement, couplé à un système de
mémoïsation on-disk offre de nombreux avantages. On peut notamment citer :

\begin{description}
  \item[Reproductibilité] Cette spécification déclarative de la configuration
    permet de ne pas dépendre d'un état antérieur du système.
    De plus, chaque ``dérivation'' (l'équivalent Nix des paquets des
    gestionnaires de paquets traditionnels) est instanciée dans un
    environnement isolé avec uniquement ses dépendances disponibles, ce qui
    assure l'absence de dépendance implicite envers d'autres parties du système.

    Une reproductibilité totale n'est malheureusement pas possible de façon
    réaliste car il reste toujours des paramètres variables (la configuration
    physique de la machine en particulier, l'horloge, l'état d'entropie du
    système etc..) qui peuvent influer sur la compilation des logiciels. Nix
    essaie au mieux de mitiger ces sources de non reproductibilité, par exemple
    en patchant le compilateur c pour que des macros comme \verb|__TIME__| ou
    \verb|__DATE__| (qui sont normalement remplacées par l'heure et la date de
    compilation) renvoient toujours la même valeur.

  \item[Possibilité de rollback] De par le fonctionnement du logiciel, une mise
    à jour n'est pas destructive mais ajoute juste une nouvelle configuration à
    côté des précédentes (et fait pointer le système vers cette configuration).
    Une conséquence est qu'il est possible de revenir en arrière de façon
    transparente si besoin est.
    De plus, la modularité du système fait que seules les parties du système
    qui sont effectivement mises à jour (logiciels mis à jour et leurs
    dépendances, configurations modifiées) sont dupliqués sur le disque, ce qui
    rend ce système assez peu coûteux en espace disque.

  \item[Mises à jour atomiques] Une mise à jour pour Nix consiste à réaliser la
    nouvelle configuration du système, et ensuite à faire pointer le système
    effectif vers cette configuration (ce qui correspond essentiellement à
    modifier un lien symbolique). L'opération de mise à jour en tant que
    telle est donc quasiment atomique, ce qui réduit à zéro les risques qu'une
    mise à jour interrompue casse le système.

  \item[Environnement locaux] Le fonctionnement de Nix permet trivialement
    d'installer des paquets avec une visibilité plus réduite que l'ensemble du
    système. Ce peut être par exemple une installation locale à un utilisateur
    − il est possible dans nix d'autoriser aux utilisateurs d'installer (sans
    accès root) des logiciels dans leur profile personnel sans pour autant
    devoir dupliquer leur installation.

    Plus encore, il est possible d'installer quelques logiciels de façon plus
    locale encore, accessibles par exemple uniquement dans un certain shell à
    la manière des virtualenvs python. Cette propriété est extrêmement pratique
    parce qu'elle permet par exemple de configurer en une seule commande
    l'environnement nécessaire pour modifier un logiciel pourvu que celui-ci
    fournisse un fichier décrivant ses dépendances.
\end{description}

\paragraph{Le langage Nix :}

Une grande partie de système repose donc sur le langage utilisé pour décrire le
système.

Ce langage est essentiellement un lambda-calcul (évalué paresseusement), avec
des listes et des enregistrements, ainsi qu'une notion de types à l'exécution
(via des fonctions comme \lstinline{isInt} qui renvoient \lstinline{true} si et
seulement si leur argument est un entier).
Il est de plus non typé (pas tant par choix que par manque de temps lors de sa
conception).

Ajouter un système de types à ce langage a deux objectifs :

\begin{itemize}
  \item \texttt{nixpkgs}, le répertoire de paquets Nix comporte aujourd'hui
    plusieurs centaine de milliers de lignes de code, avec une structure
    parfois très complexe.
    L'absence de typage commence à rendre toute modification non triviale sur
    ce dépôt très délicate et dangereuse.
  \item Nix explore une piste extrêmement intéressante en appliquant à la
    gestion de systèmes des principes issus de la programmation. Ajouter un
    système de types ouvre une nouvelle piste extrêmement riche à cette
    recherche. Le système de type présenté ici reste strictement cantonné au
    langage et n'interagit pas avec la partie ``gestion des paquets'' en tant que
    telle, mais offre une base à une telle extension.
\end{itemize}

Le langage présente de nombreuses caractéristiques contraignant les
possibilités de typage :

\begin{itemize}
  \item La présence de types à l'exécution nécessite de disposer de types union
  \item Les enregistrements peuvent avoir leurs étiquettes définis
    dynamiquement (l'étiquette n'est pas nécessairement une chaine de
    caractères, mais peut être le résultat de l'évaluation d'une expression
    arbitraire − dans la mesure où elle s'évalue vers une chaîne de
    caractères).
  \item Le langage existe depuis dix ans et a évolué en l'absence de système de
    types. En conséquence, de nombreuses constructions idiomatiques sont
    difficiles voire impossibles à typer.
\end{itemize}


\subsubsection{Syntaxe et sémantique}
% Expliquer au passage les points problématiques pour le typage
La syntaxe complète du langage est donnée par la figure \pref{nix::syntax}.
Sa sémantique est définie informellement ci dessous (une sémantique complète
est donnée en section \ref{sec:nix-light}).

Le langage est un lambda-calcul, avec

\begin{description}
  \item[Des constantes de base] (entiers, chaînes de caractères, booléens et
    paths représentant des chemins d'accès Unix).

  \item[Des let-bindings] qui sont récursifs par défaut.

  \item[Des listes] définies par la syntaxe \lstinline{[ <expr> ... <expr> ]}.

  \item[Des enregistrements] définis par la syntaxe
    \lstinline|{ <record-field>; ... <record-field>; }|. % chktex 26

    Les étiquettes des champs peuvent être le résultat d'expressions
    arbitraires − pourvu que ces expressions s'évaluent en une chaîne de
    caractères.

    Ces enregistrements peuvent être définis récursivement (avec le mot clé
    \lstinline{rec}), auquel cas les champs peuvent dépendre les un des autres.

    Par exemple, l'expression

    \begin{lstlisting}
    rec {
      x = 1;
      y = x;
    }
    \end{lstlisting}
    est équivalente à
    \begin{lstlisting}
    {
      x = 1;
      y = 1;
    }
    \end{lstlisting}

  \item[Des if-then-else].

  \item[Une syntaxe d'accès aux chants des records] de la forme
    \lstinline{<expr>.<access-path>}
    Comme pour a définition d'un record, le nom des champs peut être une
    expression arbitraire.

    Si le champ n'est pas présent dans le record, l'évaluation est stoppée.

    Une valeur par défaut en cas de champ absent peut être donnée avec la
    syntaxe \lstinline{<expr>.<access-path> or <expr>}.

    Par exemle, \lstinline|{ x = { y = 1; }; }.x.y| s'évalue en \lstinline{1},
    \lstinline|{ x = { y = 1; }; }.x.z or 2| s'évalue en \lstinline{2} et
    \lstinline|{ x = { y = 2; }; }.y| renvoie une erreur.

  \item[Des patterns] (uniquement pour les lambdas).

    Les seuls patterns non-triviaux existant sont les patterns reconaissant un
    enregistrement, de la forme
    \lstinline|{ <pattern-field>, ..., <pattern-field> }| % chktex 26
    ou
    \lstinline|{ <pattern-field>, ..., <pattern-field>, "..." }|. % chktex 26 chktex 18

    Le motif <pattern-field> est de la forme \lstinline{<ident>} ou
    \lstinline{<ident> ? <expr>} pour spécifier une valeur par défaut dans le
    cas où le champ est absent.

    Contrairement à la pluspart des langages où la variable de capture est
    indépendante du nom du champ (par exemple en OCaml où un pattern
    reconaissant un enregistrements serait de la forme
    °{ x = fieldname; y = otherfieldname }°, le motif °{ x; y }° n'étant qu'un
    sucre syntaxique pour °{ x = x; y = y }°), Nix impose que les noms des
    champs soient les mêmes que ceux des variables de capture.

  \item[Des opérateurs infixes] comme °+°, °-°, etc..

\end{description}

\begin{figure}
  \def\dots{$\cdots$}
  \begin{grammar}
    \bfseries
    <expr> ::=
    <ident> | <constant>
    \alt $\lambda$ <pattern>.<expr> | <expr> <expr>
    \alt let <ident> = <expr>; \dots; <ident> = <expr>; in <expr>
    \alt [ <expr> \dots <expr> ]
    \alt \{ <record-field>; \dots; <record-field>; \}
    \alt rec \{ <record-field>; \dots; <record-field>; \}
    \alt if <expr> then <expr> else <expr>
    \alt <expr>.<acces-path>
    \alt <expr>.<acces-path> or <expr>
    \alt <expr> <infix-op> <expr>

    <constant> ::= <string> | <integer> | <boolean> | <paths>

    <record-field> ::= inherit <ident> \dots <ident>
    \alt inherit (<expr>) <ident> <ident>
    \alt <ident> = <expr> | \$\{ <expr> \} = <expr>

    <pattern> ::= <record-pattern> | <record-pattern>@<ident> | <ident>

    <record-pattern> ::= \{ <record-pattern-field>, \dots, <record-pattern-field> \}
    \alt \{ <record-pattern-field>, \dots, <record-pattern-field>, \ldots \}

    <record-pattern-field> ::= <ident> | <ident>? <expr>

    <access-path> ::= <access-path-item>. \dots . <access-path-item>

    <access-path-item> ::= <ident> | \$\{ <expr> \}

    <infix-op> ::= + | - | * | / | // | ++ | \dots
  \end{grammar}
  \caption{Syntaxe du langage Nix\label{nix::syntax}}
\end{figure}




\subsection{Types ensemblistes}
% Présentation de l'interprétation ensembliste des types
% Justification informelle de pourquoi le système convient à Nix
Les caractéristiques du langage imposent plusieurs restrictions sur le système
de type. En particulier :

\begin{enumerate}
  \item Du fait de la grande base de code (non typé) existant, le système doit
    être assez souple pour accepter un maximum d'expressions, sans sacrifier
    les garanties de sécurité pour autant.

  \item Il doit pouvoir gérer des enregistrements anonymes avec des labels
    définis dynamiquement, ainsi que différentes opérations sur ces records
    (ajout et suppression de champs, fusion d'enregistrements, extraction de
    l'ensemble des labels, etc..).

  \item Les prédicats sur les types imposent de disposer de types union, ainsi
    que d'une opération de typecase.
\end{enumerate}

Plusieurs systèmes de types existants répondent aux deux derniers critères. Le
système choisi est basé sur les travaux d'Alain Frish sur les types
ensemblistes \cite{Fri04}, avec les extensions apportées par Kim
Nguyễn \cite{phdkim} et Giuseppe Castagna et Victor Lanvin \cite{CL17}.

Ce système se base sur une interprétation de la sémantique des types comme un
ensemble de valeurs, ce qui permet une interprétation naturelle des opérations
d'union, intersection ou différence comme les opérations correspondantes sur
les ensembles sous-jacents.
Cette interprétation permet aussi la définition d'une relation de sous-typage
correspondant naturellement à la relation d'inclusion sur les interprétations
ensemblistes.
Toutes les opérations mentionnées ci-dessus ainsi que la relation de
sous-typage sont décidables dans ce système.

Les travaux de G. Castagna et V. Lanvin (\cite{CL17}) ajoutent à ce système un
type graduel, ce qui permet de répondre à la première problématique ci-dessus.
L'ajout de ce type graduel est malheureusement pour l'instant au détriment du
polymorphisme, mais une récente extension \cite{Call18} ouvre à ce sujet de
nouvelles perspectives.


\section{Nix-light} % TODO: find another name for this
\label{sec:nix-light}

\subsection{Motivation}
% Explication de pourquoi nix est trop permissif et pourquoi il vaut mieux
% bosser sur autre chose.

\subsection{Description}
% Description du langage, grammaire + sémantique

\subsection{De Nix à Nix-light}
% Compilation

\section{Typage}

\subsection{Types}
% Présentation des types utilisés

\subsubsection{Syntaxe}
\subsubsection{Sous-typage}
% Discussion autour du sous-typage lazy
% Sous-typage graduel

\subsection{Lambda-calcul}
% Typage du langage sans records et sans listes

\subsection{Structures de données}
% Description du typage des deux structures de données de Nix

\subsubsection{Extension des types}
% Ajout des types Cons et record.

\subsubsection{Listes}
% Typage des listes. Rien de très compliqué, mais les regexp-lists nécessitent
% peut-être un peu d'explication. À voir si on garde comme une sous-partie ou
% si on merge dans la section "lambda-calcul", vu que c'est ni central ni
% original (mais joli par contre)

\subsubsection{Records}
% Typage des records. Probablement plein de choses à dire ici.

\section{Soundness du typage}
% Difficulté de définir la soundness avec le type graduel
% Blablater sur la difficulté des preuves.

\section{Implémentation}
% Tout ce qui concerne l'implémentation. Probablement des choses à dire

\begin{appendices}
  \section{Preuves de typage}
\end{appendices}

\bibliographystyle{alpha}
\bibliography{references}
\end{document}
