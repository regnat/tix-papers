Les caractéristiques du langage imposent plusieurs restrictions sur le système
de type. En particulier :

\begin{enumerate}
  \item Du fait de la grande base de code (non typé) existant, le système doit
    être assez souple pour accepter un maximum d'expressions, sans sacrifier
    les garanties de sécurité pour autant.

  \item Il doit pouvoir gérer des enregistrements anonymes avec des labels
    définis dynamiquement, ainsi que différentes opérations sur ces records
    (ajout et suppression de champs, fusion d'enregistrements, extraction de
    l'ensemble des labels, etc..).

  \item Les prédicats sur les types imposent de disposer de types union, ainsi
    que d'une opération de typecase.
\end{enumerate}

Plusieurs systèmes de types existants répondent aux deux derniers critères. Le
système choisi est basé sur les travaux d'Alain Frish sur les types
ensemblistes \cite{Fri04}, avec les extensions apportées par Kim
Nguyễn \cite{phdkim} et Guiseppe Castagna et Victor Lanvin \cite{CL16}.

Ce système se base sur une interprétation de la sémantique des types comme un
ensemble de valeurs, ce qui permet une interprétation naturelle des opérations
d'union, intersection ou différence comme les opérations correspondantes sur
les ensembles sous-jacents.
Cette interprétation permet aussi la définition d'une relation de sous-typage
correspondant naturellement à la relation d'inclusion sur les interprétations
ensemblistes.
Toutes les opérations mentionnées ci-dessus ainsi que la relation de
sous-typage sont décidables dans ce système.

Les travaux de G. Castagna et V. Lanvin (\cite{CL16}) ajoutent à ce système un
type graduel, ce qui permet de répondre à la première problématique ci-dessus.
L'ajout de ce type graduel est malheureusement pour l'instant au détriment du
polymorphisme, mais une récente extension \cite{CL17} ouvre à ce sujet de
nouvelles perspectives.
