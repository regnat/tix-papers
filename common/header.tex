\documentclass{article}

% \usepackage[margin=1in]{geometry}

% Some color
% \usepackage[usenames,dvipsnames,svgnames,table]{xcolor}

% And hypertext links everywhere
\usepackage{hyperref}

% To get a proper text encoding with xelatex
\usepackage{fontspec}

% For multiline captions
\usepackage{caption}

% Some math packages
\usepackage{amssymb}
\usepackage{amsmath}
\usepackage{bm}
\usepackage{stmaryrd}
\usepackage{mathtools}

% Render inference rules
\usepackage{mathpartir}
% The labels of those rules
\newcommand{\irlabel}[1]{\text{\emph{(#1)}}}

% Used to display stuff like x/y or x//y
\usepackage{xfrac}

% Fore code blocks
\usepackage{listings}
\lstdefinelanguage{NLight}{%
  morekeywords={%
    let,in,if,then,else,Cons,Nil
  },%
  morekeywords={[2]Int,true,false,Bool,T},   % types go here
  otherkeywords={:,=[]}, % operators go here
  literate={% replace strings with symbols
    {->}{{$\to$}}{2}
    {lambda}{{$\lambda$}}{1}
    {tin}{{$\in$}}{2}
    {\&}{{$\wedge$}}{2}
  },
  basicstyle={\sffamily},
  keywordstyle={\bfseries},
  keywordstyle={[2]\itshape}, % style for types
  keepspaces,
  mathescape % optional
}[keywords,comments,strings]
\lstset{%
  escapeinside={//*}{*//},
  breaklines=true,
  mathescape=true,
  language=NLight
}

\usepackage{todo}

% Tweak the syntax
\usepackage{syntax}
\renewcommand{\grammarlabel}[2]{\meta{#1 #2}}
\newcommand{\meta}[1]{\ensuremath{#1}} % For meta syntax
\renewcommand{\|}{\textrm{|}}

\newcommand{\assign}[2]{\ensuremath{\sfrac{#2}{#1}}}
\newcommand{\assignp} [2] {\assign{#1}{#2}}
\newcommand{\subst} [3] {#3 [\assign{#1}{#2}]}
\newcommand{\substp} [3] {#3 [\assignp{#1}{#2}]}
\newcommand{\dstep} [2] {#1 \ensuremath{\rightarrow} #2}
\newcommand{\ndstep} [2] {#1 \ensuremath{\nrightarrow} #2}
\newcommand{\ndsteps} [2] {#1 \ensuremath{\nrightarrow^*} #2}
\newcommand{\dstepa} [3] {\dstep{#1}{&#2}~\emph{#3} \\}
\newcommand{\eqdef}[2]{#1 \ensuremath{\overset{\text{def}}{=}} #2}
\newcommand{\eqdefa}[3]{\eqdef{#1}{&#2} \emph{#3} \\}

% TODO remove this, it's ugly
\newcommand{\xone}{\ensuremath{x_1}}
\newcommand{\xn}{\ensuremath{x_n}}
\newcommand{\eone}{\ensuremath{e_1}}
\newcommand{\etwo}{\ensuremath{e_2}}
\newcommand{\en}{\ensuremath{e_n}}
% TODO: redefine to get the 0 and 1 of types (with double bar)
\newcommand{\zero}{0}
\newcommand{\one}{1}

% Rendering of symbols and operators
\newcommand{\discrete}[2]{\left\{ #1,\cdots, #2 \right\}}
\newcommand{\ty}[1]{\texttt{#1}}
\newcommand{\set}[1]{\ensuremath{\mathcal{#1}}}
\newcommand{\quasiconst}[1]{\overset{#1}{\twoheadrightarrow}}
\DeclareMathOperator\dom{dom}
\DeclareMathOperator\deff{def}
\DeclareMathOperator\var{\mathcal{V}}
\DeclareMathOperator\A{\mathcal{A}}
\DeclareMathOperator\Bt{\mathcal{B}}
\newcommand{\orthsum}{\oplus^\bot}
\newcommand{\orthplus}{\diamond}
\newcommand{\subtype}{\leq}
\newcommand{\subtypeG}{\precapprox}
\newcommand{\tinfer}{\vdash^\Uparrow}
\newcommand{\tcheck}{\vdash^\Downarrow}
\newcommand{\tIC}{\vdash^\delta}
\newcommand{\onerec}{\{ \textbf{..} \}}
\DeclareCollectionInstance{plainmath}{xfrac}{mathdefault}{math}
{%
  slash-symbol = \sslash{}
}
\newcommand{\ofTypeP}[2]{\UseCollection{xfrac}{plainmath}\sfrac{#1}{#2}}

\newcommand{\Γ}{\Gamma}
\newcommand{\τ}{\ensuremath{\tau}}
\newcommand{\σ}{\sigma}
\DeclareMathOperator\any{\textsc{Any}}
\DeclareMathOperator\grad{\star}
\newcommand{\undef}{\nabla}
\DeclareMathOperator\Int{Int}
\DeclareMathOperator\Bool{Bool}
\newcommand{\λ}{\lambda}
\newcommand{\recleq}{\sqsubsetleq}

% Easier writing of references
\newcommand{\pref}[1]{\ref{#1} at page~\pageref{#1}}

\date{}

