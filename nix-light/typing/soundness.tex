We have the two classical results of \emph{Subject reduction} and
\emph{Progress} which entail type soundess for the non-gradual part of the
language.
Note that because we defined two typing judgements, both result will be stated
twice.

We first prove the following lemmas:

\begin{lemma}\label{lemma:inferCheck}
  Let $e$ be an expression, $\Γ$ a typing environment and $\τ$ and $\τ'$ two
  (possibly gradual) types with $\τ \subtypeG \τ'$.
  If $\Γ \tinfer e : \τ$ then $\Γ \tcheck e : \τ'$.
\end{lemma}

\begin{proof}
  \todo{}
\end{proof}

\begin{lemma}[Substitution]\label{lemma:substitution}
  Let $e$ and $e'$ be expressions, $x$ be a variable, $\τ$ and $\τ'$ two types,
  $\Γ$ a typing environment and $\delta$ be $\Uparrow$ or $\Downarrow$.

  If $\Γ; x : \τ' \tIC e : \τ$ and $\Γ \tinfer e' : \τ'$ then $\Γ \tIC
  \subst{x}{e'}{e} : \τ$
\end{lemma}

\begin{proof}
  By induction on the typing derivation of $\Γ; x : \τ' \tIC e : \τ$. We
  replace every Ivar rule introducing $\Γ ; x : \τ' \tinfer x : \τ'$ by a
  derivation of $\Γ \tinfer e' : \τ'$, and every Cvar rule introducing $\Γ ; x
  : \τ' \tcheck x : \τ''$ (with $ \τ' \subtypeG \τ'')$ by a derivation of $\Γ
  \tcheck e' : \τ''$ (which exists because of Lemma~\ref{lemma:inferCheck}).

  This builds a new derivation of $\Γ \tinfer \subst{x}{e'}{e} : \τ$.
\end{proof}

\begin{theorem}[Subject reduction $\Uparrow$]\label{thm:subj-reduction-infer}
  For any pair $e, e'$ of terms (of nix-light), if $\Γ \tinfer e : t$ and $e
  \rightarrow e'$, then $\Γ \tinfer e' : t$.
\end{theorem}

\begin{proof}
  We consider an expression $e$ such that $\Γ \tinfer e : t$.
  We prove by induction on the derivation of $\Γ \tinfer e : t$ that $\forall
  e', (e \rightarrow e') \Rightarrow (\Γ \tinfer e' : t)$.

  Let's consider the various possibilities for the last rule of the derivation
  $\Γ \tinfer e : t$.

  \begin{description}
    \item[IVar,IConst,IAbs,ICons] The expression $e$ is a value and can't be
      reduced, so the property holds.
    \item[IApp]
      $\inferrule{%
        \Γ \tinfer e_1 : t_1 \\ \Γ \tinfer e_2 : t_2 \\
        t_1 \subtypeG \zero \rightarrow \one \\
        t_2 \subtypeG \dom(t_1)
      }{%
        \Γ \tinfer e_1~e_2 : t_1 \circ t_2
      }$

      The expression $e$ has then the form $e_1~e_2$ with $\Γ \tinfer e_1
      : t_1$ and $\Γ \tinfer e_2 : t_2$ (and $t = t_1 \circ t_2$).
      It can be reduced in three different ways (depending of the form of $e_1$
      and $e_2$):
      \begin{itemize}
        \item If $e_1$ is a value $\λ r. e_0$, then the only possible reduction
          is by applying the $\beta$-reduction rule, so the only choice for
          $e'$ is $\subst{x}{e_2}{e_0}$ (where $x = \var(r)$).

          Moreover, a case analysis on the different typing rules shows that
          the last rule of the derivation of $\Γ \tinfer \λ r . e_0 : t_1$ can
          only be the Iabs rule:
          \[
            \inferrule{%
              x:t_x \vdash r:t_x \\ \Γ; x:t_x \tinfer e_0:s
            }{%
              \Γ \tinfer \λ r.e_0 : t_x \rightarrow s
            }
          \]
          So $t_1$ has the form $t_x \rightarrow s$. This means that $t_2$ is a
          subtype of $t_x = \dom(t_x \rightarrow s)$ and that $t'$ is equal to
          $s = (t_x \rightarrow s) \circ t_2$.

          Moreover, as $\Γ; x:t_x \tinfer e_0:s$ and $\Γ \tinfer e_2 : t_2$
          with $t_2 \subtypeG t_x$, the Lemma~\ref{lemma:substitution} allows
          us to conclude that $\Γ \tinfer e' : t'$.
        \item If $e_1$ is a value $\λ p.e_0$ (with $p$ not in the form $x$ or
          $x:t$), and $e_2$ is a value, then we can use the same reasoning.
        \item In the other cases, we can reduce either $e_1$, either $e_2$.

          If we reduce $e_1$ to $e'_1$, we get a new expression $e' =
          e'_1~e_2$, and $e'_1$ satisfies $\Γ \tinfer e'_1 : t_1$ (by induction
          hypothesis). By re-applying the Iapp rule, we get $\Γ \tinfer
          e'_1~e_2 : t$, thus $\Γ \tinfer e' : t$.

          The same holds if we reduce $e_2$.

      \end{itemize}
      \item[Let]
        $\inferrule{%
          \Γ; x_1 : \any \tinfer e_1 : t_1\\
          \Γ ; x_1 : t_1 \tinfer e_2 : t
        }{%
          \Γ \tinfer \text{let } e_1 = x_1 \text{ in } e_2 : t
        }$

        The expression $e$ has the form ``let $e_1 = x_1$ in $e_2$''.
        It reduces (and may only reduce to) $e' = \subst{x_1}{e_1}{e_2}$.
        By applying the substitiution lemma, we can deduce that $e'$ has type
        $t$.
      \item[LetAnnot]
        The subject reduction property do not hold for the \textbf{LetAnnot}
        case for now.~\todo{Make this hold}
      \item[ITcase]
        $\inferrule{%
          \Γ \tinfer e_0 : t_0 \\
          t_0 \not\subtype s \Rightarrow \Γ; x : t_0 \wedge \lnot s \tinfer e_2 : t_2 \\
          t_0 \not\subtype \lnot s \Rightarrow \Γ; x : t_0 \wedge s \tinfer e_1 : t_1 \\
        }{%
          \Γ \tinfer (x = e_0 \in s) ? e_1 : e_2 : t_1 \vee t_2
        }$

        If $e_0$ is not a value, then the only possible reduction for $e$ is to
        reduce it to an expression $e'_0$ which will have a type $t'_0 \subtype
        t_0$ under the context $\Γ$ (by induction hypothesis).
        By re-applying the \textbf{ITcase} rule, we obtain a new type $t'$ for
        the expression $(x = e'_0 \in s) ? e_1 : e_2$, which is a subtype of
        $t$.

        If $e_0$ is a value $v$, then the only possible reductions are if
        $\tinfer v : s$ or $\tinfer v : \lnot s$\todo{Current typing rules
        do not allow asserting that this is always true, which will b a deal
        broker for Progress, see~\cite{Fri08} for a way to enforce this}.

        Assume the first one. The expression $e$ then reduces to
        $\subst{x}{v}{e_1}$, and (by application of the substitution Lemma), we
        know that $\Γ \tinfer \subst{x}{v}{e_1} : t_1$. As $t_1 \subtype t$
        (because $t = t_1 \vee t_2$), we get the expected result.

        By symmetry, this also holds if $\tinfer v : \lnot s$.
  \end{description}
\end{proof}

\begin{theorem}[Subject reduction $\Downarrow$]\label{thm:subj-reduction-check}
  For any pair $e, e'$ of terms (of nix-light), if $\Γ \tcheck e : t$ and $e
  \rightarrow e'$, then $\Γ \tcheck e' : t$.
\end{theorem}

\begin{proof}
  The proof is similar to the one of~\autoref{thm:subj-reduction-infer}.
\end{proof}
