La grammaire de Nix-light est donnée en figure \pref{nix-light::grammar}.

Ce langage reprend essentiellement toutes les caractéristiques de Nix, mais en
rendant syntaxiquement reconnaissable les éléments qui nécessitent un traitement
spécial au typage.

Ainsi, le if-then-else est remplacé par le plus général « typecase » de la
forme `(x = e0 tin t) ? e1 : e2` (qui s'évalue en `e1` si `e0` s'évalue en une
value de type `t` et en `e2` sinon). Le cas général `if e0 then e1 else e2`
peut ainsi être compilé en `(x = e0 tin true) ? e1 : e2` (ou `x` n'apparait pas
libre dans `e1` et `e2`), alors qu'un cas particulier comme
`if isInt x then e1 else e2` sera compilé vers `(x = x tin Int) ? e1 : e2`.
Cela permet de reléguer la reconnaissance des éléments particuliers à une phase
de compilation préalable au typage, et permet du coup d'alléger le système de
types. En contrepartie, cela diminue la puissance du système de types,
puisqu'une expression comme `let f = isInt; in if f x then x else 1` ne pourra
pas être reconnue par le compilateur dans la mesure où celui-ci n'a pas
d'information de type pour savoir que `f` doit être considérée comme un
prédicat sur les types.

\begin{figure}
  \begin{lstlisting}
    <e> ::=
        <x> | <c>
      | <e>.<a> | <e>.<a> or <e>
      | $\lambda$<p>.<e> | <e> <e>
      | let <vr> = <e>; $\cdots{}$; <vr> = <e>; in <e>
      | Cons (<e>, <e>)
      | (<x> = <e> $\in$ <t>) ? <e> : <e>

    <c> ::= <s> | <i> | <b> | Nil

    <p> ::= <rp> | <rp>@<x> | <vr>

    <rp> ::= <rp>:τ
      | { <rpf>, $\cdots$, <rpf> }
      | { <rpf>, $\cdots$, <rpf>, ... }

    <rpf> ::= <vr> | <vr> ? <c>

    <vr> ::= <x> | <x>:<τ>

    <t> ::= <c> | <t> $\rightarrow$ <t>
      | <t> $\vee$ <t> | <t> $\wedge$ <t> | $\lnot$ <t>
      | [<R>]
      | bool | int | string

    <r> ::= <t> | <r>+ | <r>* | <r>?
      | <r> <r> | <r> ¦ <r>

    <τ> ::= <c> | <τ> $\rightarrow$ <τ>
      | <τ> $\vee$ <τ> | <τ> $\wedge$ <τ>
      | [<R>]
      | bool | int | string | ?

    <ρ> ::= <τ> | <ρ>+ | <ρ>* | <ρ>?
      | <ρ> <ρ> | <ρ> ¦ <ρ>
  \end{lstlisting}
  \caption{Grammaire de Nix-light\label{nix-light::grammar}}
\end{figure}
